\documentclass[compress]{beamer}
\usepackage{default}
\usepackage{graphicx}
\usepackage{amsfonts}
\usepackage{amssymb}
\usepackage{amsmath}
\usepackage[brazil]{babel}
\usepackage[utf8]{inputenc}
\usetheme{Szeged}
\usecolortheme{whale}

\title{Registo de Imagens\\Algoritmo Demons}
\author{Thiago de Gouveia Nunes}
\date{}

\begin{document}

\frame{\titlepage}

\begin{frame}
  O algoritmo Demons foi desenvolvido utilizando o método de fluxo óptico. Esse métoo é usado para encontrar um campo vetorial de deslocamento que leva
  um volume móvel até outro volume estático. \\
  A principal hipótese desse método é que as intensidades dos dois volumes são iguais.
\end{frame}

\begin{frame}
  Partindo da hipótese acima, o campo vetorial é dado por: \\
  \begin{equation*}
    \overrightarrow{v_i} . \overrightarrow{\triangledown}s_i = M(i) - S(i)
  \end{equation*}
\end{frame}

\begin{frame}
  A equação acima tem váriaveis demais, e não pode ser resolvida. Para resolver esse problema, o demons foi desenvolvido usando um processo
  iterativo. A cada passo um campo novo é calculado, e esse campo é aplicado a um novo volume, chamado de deformado, até que o campo se estabilize.
\end{frame}

\begin{frame}
  Algumas variantes do algoritmo:
  \begin{itemize}
    \item Cálculo do campo usando as informações dos dois volumes. \\
    \item Introdução de um valor para ajuste iterativo do campo. \\
    \item Utilização da intensidade do volume móvel ao invés do estático.
  \end{itemize}
\end{frame}

\begin{frame}
  Podemos modificar o critério de parada também. Definindo uma norma relativa, que representa o incremento da iteração atual no campo, podemos parar o
  algoritmo quando esse incremento é baixo.
\end{frame}

\end{document}
